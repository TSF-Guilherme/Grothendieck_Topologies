\documentclass{article}
\usepackage{babel}                              % Hifenização em português
\usepackage[utf8]{inputenc}                     % Acentuação com o teclado
\usepackage{amsmath, amssymb, exscale, amsthm}  % Math coisas que vc precisa sempre
\usepackage{indentfirst}                        % Identa todo início de seção
\usepackage{comment}                            % Permite usar o \begin{comment}
\usepackage{geometry}                           % Permite usar o \geometry{} para definir as margens da página
\usepackage{tikz-cd}                            % Desenhar diagramas comutativos. Por algum motivo a opção "brazil" do pacote "babel" interfere na hora de colocar nome nas setas
\usepackage{tcolorbox}

%\usepackage{makeidx}   % Isso serve para criar indices remissivos, favor usar \makeindex exatamente uma linha antes de \begin{document}
%\usepackage{graphicx}  % Inserir imagem
%\usepackage{float}     % Permite vc botar [H] ao inserir a imagem i.e. "PUT THIS SHIT EXACTLY HERE MF
%\usepackage{fancyhdr}  % Permite usar o \pagestyle{fancy}
%\usepackage{extsizes}  % Dá mais opções de tamanho de fonte
%\usepackage{bbm}       % Permite usar \mathbbm{} (sinceramente, prefiro \mathbb{})
%\usepackage{setspace}  % Permite mudar o espaçamento entre palavras, ex: \onehalfspace
%\usepackage{pifont}    % Serve pra colocar simbolos bonitinhos com \ding{} 

\parskip = 3mm                                          % Espaçamento entre paragrafos 
\geometry{bottom=3.5cm, top=3cm, left=4cm, right=4cm}   % Define margens
\setlength{\parindent}{0pt}                             % Define a identação dos parágrafos. Setamos 0 e criamos um \newcommand para adicionar indentação em parágrafos específicos 
\setlength{\jot}{10pt}                                  % Espaçamento entre linhas no ambiente \align ou \eqnarray 
\linespread{1.3}                                        % Espaçamento entre linhas
\newtcolorbox{adendo}[1][]{
    colback=blue!5!white,                                       % fundo azul clarinho
    colframe=blue!50!black,                                     % borda azul escura
    coltitle=white,                                             % cor do título
    fonttitle=\bfseries,                                        % título em negrito
    title=#1,                                                   % O título será definido no corpo do texto no 1º colchetes
    boxrule=1pt,                                                % espessura da borda
    arc=3mm,                                                    % cantos arredondados
    top=2mm, bottom=2mm,                                        % espaçamento interno
    width=\dimexpr\textwidth+1cm\relax,                         % <<< maior que o textou 
    left=3mm, right=3mm,                                        % margem interna extra
    enlarge left by=-5mm,                                       % puxa pra fora da margem
    enlarge right by=-5mm                                       % idem
}

%\pagestyle{}                       % Define estilo da página. Opções: "empty" "plain" "headings" "fancy"(com o package) etc.


% Isso aqui não é muito útil para listas de exercícios, mas permitirá vc usar \begin{theorem} sempre que quiser escrever um teorema.

\theoremstyle{plain}                    % Define o estilo da fonte no \begin{theorem} para os \newtheorem que seguem abaixo desta linha
\newtheorem{theorem}{Teorema}
\newtheorem{lemma}{Lema}
\newtheorem*{proposition}{Proposição}
\newtheorem{corollary}{Corolário}
\theoremstyle{definition}               
\newtheorem*{definition}{Definição}
\newtheorem{example}{Exemplo}
\theoremstyle{remark}
\newtheorem*{remark}{Observação}
\newtheorem*{notation}{Notação}

\newcommand{\Gl}{\mathrm{GL}}
\newcommand{\Mat}{\mathrm{Mat}}
\newcommand{\Or}{\mathrm{O}}
\newcommand{\SO}{\mathrm{SO}}
\newcommand{\U}{\mathrm{U}}
\newcommand{\SU}{\mathrm{SU}}
\newcommand{\N}{\mathbb{N}}
\newcommand{\Z}{\mathbb{Z}}
\newcommand{\Q}{\mathbb{Q}}
\newcommand{\R}{\mathbb{R}}
\newcommand{\C}{\mathbb{C}}
\newcommand{\cont}{\subseteq}                   % Contido como deve ser
\newcommand{\Fi}{\varphi}                       % phi bonito (o comando \fi já existe e não sei o que é)
\newcommand{\act}{\circlearrowleft}             % Ação de grupo
\newcommand{\Ktil}{\Tilde{K}}                   % Um til em cima das letras
\newcommand{\Btil}{\Tilde{B}}
\newcommand{\dlim}{\displaystyle\lim}           % Limite bonito
\newcommand{\e}{\varepsilon}                    % Epsilon bonito
\newcommand{\ind}{\setlength{\parindent}{15pt}} % Use isso na primeira palavra de um parágrafo para indentar Ex: {\ind "Não} conheço outra forma de fazer isso" 

\title{Grothendieck Topologies}
\author{Guilherme Henrique de Sá}
\date{}

\begin{document}

\maketitle

\section*{2. Contravariant Functors}
\subsection*{2.1 Representable functors and the Yoneda Lemma}

\paragraph{Transformação Natural.}
Dado dois funtores $T, S : \mathcal{C} \to \mathcal{B}$, uma \textbf{transformação natural} é uma função que associa a cada objeto $c$ em $\mathcal{C}$ uma seta $\tau_c : S(c) \to T(c)$ em $\mathcal{B}$.

Esta associação é feita de forma que, para toda seta $f : X \to Y$ em $\mathcal{C}$, valha que o seguinte diagrama comuta:

\[
\begin{tikzcd}
    X \arrow[d, "f"] & S(X) \arrow[r, "\tau_X"] \arrow[d, "S(f)"'] & T(X) \arrow[d, "T(f)"] \\
    Y & S(Y) \arrow[r, "\tau_Y"'] & T(Y)
\end{tikzcd}
\]

Ou seja, temos que: 
\[
\tau_y \circ S(f) = T(f) \circ \tau_x
\]

\begin{itemize}
    \item Transformações naturais podem ser vistas como morfismos (setas) entre funtores.
\end{itemize}

Seja $\text{Hom}(\mathcal{C}^{op}, \mathbf{Set})$ a categoria cujos elementos (objetos) são funtores contravariantes da categoria $\mathcal{C}$ para a categoria dos conjuntos, e as setas são transformações naturais.

\begin{itemize}
    \item Para cada $x$ objeto de $\mathcal{C}$, podemos definir
    \[
    h_x : \mathcal{C}^{op} \to \mathbf{Set}
    \]
    objeto de $\text{Hom}(\mathcal{C}^{op}, \mathbf{Set})$, de forma que:
    \begin{itemize}
        \item $h_x(U) = \text{Hom}(U, X)$ para cada objeto $U$ de $\mathcal{C}$;
        \item Para cada seta $\alpha : U' \to U$ em $\mathcal{C}^{op}$, temos o morfismo $h_X U \to h_X U'$ dado por composição com $\alpha$.
    \end{itemize}
    
    \item Para cada seta $f : X \to Y$ de $\mathcal{C}$, definimos uma transformação natural $h_f$ que associa a cada objeto $U$ de $\mathcal{C}^{op}$ uma seta:
    \[
    h_f(U) : h_X(U) \to h_Y(U) \quad \text{em } \mathbf{Set}
    \]
    \item Seja $\beta \in \text{Hom}(U, X)$, então
    \[
    h_f(U)(\beta) := f \circ \beta \in \text{Hom}(U, Y)
    \]
\end{itemize}

Isso define um morfismo $h_f : h_X \to h_Y$.

Para toda seta $\alpha : U' \to U$ em $\mathcal{C}^{op}$, o seguinte diagrama comuta:

\[
\begin{tikzcd}
h_X(U) \arrow[r, "h_f(U)"] \arrow[d, "h_X(\alpha)"'] & h_Y(U) \arrow[d, "h_Y(\alpha)"] \\
h_X(U') \arrow[r, "h_f(U')"'] & h_Y(U')
\end{tikzcd}
\]

\begin{itemize}
    \item Mandando cada objeto $X$ de $\mathcal{C}$ em $h_X$ e cada morfismo $f : X \to Y$ de $\mathcal{C}$ em $h_f : h_X \to h_Y$, definimos assim um funtor:
    \[
    \mathcal{C} \to \text{Hom}(\mathcal{C}^{op}, \mathbf{Set})
    \]
\end{itemize}

\paragraph{Yoneda Lemma (Versão Fraca).} Seja $x, y$ objetos de uma categoria $\mathcal{C}$. Então a função:
\[
\text{Hom}_{\mathcal{C}}(X, X) \to \text{Hom}(h_X, h_Y)
\]
que manda $f \mapsto h_f$ é bijetiva.

\medskip

\textit{(Note que isso nos diz que o funtor definido anteriormente é “plenamente fiel”, ou “cheio e fiel”).}

\paragraph{Definição.} Um \textbf{funtor representável} de uma categoria $\mathcal{C}$ é um funtor
\[
F : \mathcal{C}^{op} \to \mathbf{Set}
\]
tal que $F$ é isomorfo a $h_X$ para algum $X$ objeto de $\mathcal{C}$. Dizemos que $F$ é \textit{representado por} $X$.

\begin{itemize}
    \item Se ocorrer de $F \cong h_X$ e $F \cong h_Y$, então $h_X \cong h_Y$ e, pelo lema anterior, vai existir $f : X \to Y$ tal que $h_f : h_X \to h_Y$ é uma equivalência. Mas, pela construção de $h_f$, teremos que $f$ é uma equivalência.
\end{itemize}

\medskip

Vamos nos preparar para o lema de Yoneda (versão definitiva).

\begin{itemize}
    \item Seja $T : \mathcal{C}^{op} \to \mathbf{Set}$ um funtor e $X$ um objeto de $\mathcal{C}$. Dado $\tau : h_X \to T$, temos um morfismo $\tau_X : h_X(X) \to T(X)$.

    \item Podemos definir uma função de transformações naturais entre $h_X$ e $T$ para elementos no conjunto $T(X)$ dada por:
    \begin{eqnarray*}
        \text{Hom}(h_X, T) & \to & T(X) \\
        \tau & \mapsto & \tau_X(\text{id}_X)
    \end{eqnarray*}

    \item Dado $\xi \in T(x)$, queremos construir uma transformação natural $\tau : h_X \to T$.

    Seja $U$ um objeto de $\mathcal{C}$, então $h_X(U) = \text{Hom}(U, X)$. Um elemento $f \in h_X(U)$ é um morfismo $f : U \to X$ em $\mathcal{C}$.

    \item Definimos:
    \begin{align*}
    \tau_U : h_X(U) & \to T(U) \\ 
    f & \mapsto T(f)(\xi)
    \end{align*}

    Assim, temos uma transformação natural $\tau$ e podemos fazer $\tau_U(f) = T(f)(\xi)$.

    \item Teremos que, para todo morfismo $\alpha : U' \to U$ em $\mathcal{C}^{op}$, vale que:
    \begin{eqnarray}
        \tau_{U'}(h_X(\alpha)(f)) = \tau_{U'}(f \circ \alpha) = T(f \circ \alpha)(\xi) \\
        (T(\alpha) \circ \tau_U)(f) = T(\alpha)(T(f)(\xi)) = T(f \circ \alpha)(\xi)
    \end{eqnarray}
\end{itemize}

Como $T$ é contravariante, de (1) e (2) segue:

\[
(T(\alpha) \circ \tau_U)(f) = (T(\alpha) \circ T(f))(\xi) = T(f \circ \alpha)(\xi)
\]

\[
= (\tau_{U'} \circ h_X(\alpha))(f)
\]

Fazendo com que o diagrama comute:

\[
\begin{tikzcd}
h_X(U) \arrow[r, "\tau_U"] \arrow[d, "h_X(\alpha)"'] & T(U) \arrow[d, "T(\alpha)"] \\
h_X(U') \arrow[r, "\tau_{U'}"'] & T(U')
\end{tikzcd}
\]

Logo, $\tau$ é transformação natural.

A aplicação descrita é uma função:
\[
T(X) \to \text{Hom}(h_X, T)
\]

\paragraph{Lema de Yoneda (Versão Final).}
As funções anteriores são inversas uma da outra e vale que:
\[
T(X) \cong \text{Hom}(h_X, T)
\]

\paragraph{Observação.}
Se considerarmos $T$ representável por $Y$, i.e. $T \cong h_Y$, então:
\[
h_X(Y) = \text{Hom}(X, Y) \cong \text{Hom}(h_X, h_Y)
\]

que é a versão fraca do Lema de Yoneda.


\textbf{Def. 2.2.} Seja $ F : \mathcal{C}^{op} \to \mathbf{Set} $ um funtor.  
Um objeto universal de $F$ é um par $(X, \xi)$, onde $X$ é objeto de $\mathcal{C}$ e $\xi$ é um elemento de $FX$ tal que para cada objeto $U$ de $\mathcal{C}$ e cada $\sigma \in FU$, exista uma única seta $f : U \to X$ em $\mathcal{C}$ satisfazendo:

$$
F(f)(\xi) = \sigma
$$

Perceba que se $(X, \xi)$ é objeto universal de $F$, então a função $T$ que define a transformação natural no Lema de Yoneda é bijetiva para todo $U$.  
Assim, o morfismo $h_X \to F$ é isomorfismo.  
Segue a proposição:

\medskip

\textbf{Prop. 2.3.} Um funtor $F : \mathcal{C}^{op} \to \mathbf{Set}$ é representável se e somente se possui um objeto universal.

\begin{itemize}
    \item Se $(X, \xi)$ é objeto universal de $F$, então $F$ é representado por $X$.
    
    \item O Lema de Yoneda nos garante que $\mathcal{C}$ pode ser imerso em $\text{Hom}(\mathcal{C}^{op}, \mathbf{Set})$ e que todo funtor $F : \mathcal{C}^{op} \to \mathbf{Set}$ pode ser estendido a um funtor representável 
    $$
    h_F : \text{Hom}(\mathcal{C}^{op}, \mathbf{Set}) \to \mathbf{Set}.
    $$
\end{itemize}

Assim, vamos tratar $h_X$ como simplesmente $X$, e $\text{Hom}(h_X, F)$ como $FX$.

\section*{Exemplos:}

\begin{enumerate}
    \item Seja a categoria \(\mathbf{Set}\), onde os objetos são conjuntos e as setas são funções entre conjuntos.

    Seja \(F: \mathbf{Set}^{\text{op}} \to \mathbf{Set}\) que leva um conjunto \(S\) no conjunto \(\mathcal{P}(S)\) das partes de \(S\), e leva uma função \(f: S \to T\) em 
    \[
    Ff: \mathcal{P}(T) \to \mathcal{P}(S), \quad \sigma \mapsto f^{-1}(\sigma).
    \]

    Afirma-se que \((\{1,0\}, \{1\})\) é objeto universal de \(F\). Ora, dado um conjunto \(S\) e um subconjunto \(\sigma \in \mathcal{P}(S)\), então existe uma única \(f: S \to \{0,1\}\) tal que \(F(f)(\{1\}) = \sigma\). A função é:
    \[
    f(x) = 
    \begin{cases}
    0, & \text{se } x \notin \sigma \\
    1, & \text{se } x \in \sigma
    \end{cases}
    \]

    \item Seja \(\mathbf{HausTop}\) a categoria de todos os espaços Hausdorff com setas sendo as funções contínuas. O functor \(F: \mathbf{HausTop}^{\text{op}} \to \mathbf{Set}\), que manda um espaço \(S\) no conjunto \(FS\) dos subespaços abertos, não é representável.

    Suponha que \((X, \xi)\) é objeto universal de \(F\). Isso nos diz que \(\forall \sigma \in FS\), existe \(f: S \to X\) tal que \(F(f)(\xi) = f^{-1}(\xi) = \sigma\).

    Tomando \(\sigma = S\), então \(\xi\) possui um único elemento. Se \(\sigma = \emptyset\), então \(\# X \setminus \xi = 1\).

    Como \(X\) é Hausdorff, segue que \(X = \{a, b\}\) com a topologia discreta.

    Seja \(S\) um espaço Hausdorff tal que existe \(\sigma \in FS\) aberto mas não fechado.

    Tem que existir \(f: S \to X\) contínua tal que \(f^{-1}(\xi) = \sigma \Rightarrow f^{-1}(X \setminus \xi) = S \setminus \sigma \Rightarrow S \setminus \sigma\) é aberto e teríamos que \(\sigma\) é fechado.

    \medskip

    Assim, \((X, \xi)\) não é objeto universal!
\end{enumerate}


\begin{adendo}[1. Categories Functors and Natural Transformations]
    
\section*{5. Mônicos, Epis e Zeros}

\begin{definition}
Uma seta $e: a \to b$ é dita \textbf{invertível} se existe uma seta $e': b \to a$ na mesma categoria tal que 
$$ee' = 1_b \quad \text{e} \quad e'e = 1_a.$$
Neste caso, dizemos que $a$ e $b$ são \textbf{isomorfos} ($a \cong b$) e $e$ é dito um \textbf{isomorfismo}.
\end{definition}

\end{adendo}
\begin{adendo}
    
\begin{definition}
Um morfismo $m: a \to b$ é dito \textbf{mônico} em $\mathcal{C}$ quando, para quaisquer setas paralelas $f_1, f_2: d \to a$, se
$$m f_1 = m f_2 \quad \Rightarrow \quad f_1 = f_2.$$
\end{definition}

\begin{definition}
Uma seta $h: a \to b$ em $\mathcal{C}$ é dita \textbf{epi} se, para quaisquer setas $g_1, g_2: b \to c$, vale
$$g_1 h = g_2 h \quad \Rightarrow \quad g_1 = g_2.$$
\end{definition}

\begin{definition}
Dado $h: a \to b$, um \textbf{inverso à direita} de $h$ é uma seta $r: b \to a$ tal que $hr = 1_b$.  
$r$ é chamado \textbf{seção} de $h$.

Analogamente, um \textbf{inverso à esquerda} é uma seta $p: b \to a$ tal que $ph = 1_a$.  
$p$ é dito \textbf{retração} de $h$.
\end{definition}

\begin{proposition}
Se uma seta possui inverso à direita, então ela é necessariamente epi.  
Se possui inverso à esquerda, então é mônico.
\end{proposition}

\subsection*{Idempotentes e Splitting}

\begin{definition}
Se duas setas $g: a \to b$ e $h: b \to a$ são tais que $gh = 1_b$, então $f := hg$ está bem definido e é um \textbf{idempotente}, isto é, $f^2 = f$.
\end{definition}

\begin{definition}
Um idempotente $f$ \textbf{splita} quando existem setas $g,h$ tais que
$$f = gh \quad \text{e} \quad hg = 1_d.$$
\end{definition}

\subsection*{Objetos Especiais}

\begin{definition}
Um objeto $t$ é \textbf{terminal} quando, para todo objeto $a$, existe um único morfismo $a \to t$.
\end{definition}

\begin{definition}
Um objeto $i$ é \textbf{inicial} quando, para todo objeto $b$, existe um único morfismo $i \to b$.
\end{definition}

\begin{definition}
Um objeto que é inicial e terminal é dito \textbf{nulo}.
\end{definition}

\begin{proposition}
Objetos nulos são únicos a menos de isomorfismos e definem, para quaisquer objetos $a$ e $b$, uma seta
$$a \to z \to b,$$
chamada \textbf{seta zero} (onde $z$ é um objeto nulo).
\end{proposition}

\end{adendo}
\begin{adendo}

\subsection*{Grupoides}

\begin{definition}
Um \textbf{grupoide} é uma categoria onde toda seta é invertível.  
Um exemplo típico é o \textbf{grupoide fundamental} $\pi(X)$ de um espaço topológico $X$:  
\begin{itemize}
    \item os objetos são pontos $x \in X$;
    \item as setas $x \to x'$ são classes de homotopia de caminhos de $x$ para $x'$.
\end{itemize}
\end{definition}

\begin{proposition}
Se $G$ é um grupoide e $x$ é um objeto de $G$, então $\text{Hom}_G(x,x)$ forma um grupo.
\end{proposition}

\begin{proposition}
Se existe morfismo ligando dois objetos $x \to x'$ em $G$, então os grupos $\text{Hom}_G(x,x)$ e $\text{Hom}_G(x',x')$ são isomorfos (pela conjugação).
\end{proposition}

\begin{definition}
Um grupoide é dito \textbf{conexo} se existe uma seta ligando quaisquer dois objetos.  
Um grupoide conectado é caracterizado pelo seu conjunto de objetos e um grupo $\text{Hom}_G(x,x)$.
\end{definition}

\begin{remark}
No caso do grupoide fundamental $\pi(X)$, o conjunto é $X$ e o grupo $\text{Hom}(x,x)$ é dito \textbf{grupo fundamental}.
\end{remark}

\end{adendo}

\end{document}
